\documentclass[letterpaper]{article} 
\usepackage[utf8]{inputenc}
\usepackage[T1]{fontenc}
\usepackage{amsmath}
\usepackage{amsfonts}
\usepackage{amssymb}
\usepackage{array}
\usepackage{booktabs}
\usepackage{multirow}
\usepackage{hyperref}
\usepackage[version=4]{mhchem}
\usepackage{stmaryrd}
\usepackage[dvipsnames]{xcolor}
\colorlet{LightRubineRed}{RubineRed!70}
\colorlet{Mycolor1}{green!10!orange}
\definecolor{Mycolor2}{HTML}{00F9DE}
\usepackage{graphicx}
\usepackage{amsmath}
\usepackage{graphicx}
\usepackage{capt-of}
\usepackage{lipsum}
\usepackage{algpseudocode}
\usepackage{fancyvrb}
\usepackage{tabularx}
\usepackage{listings}
\usepackage[export]{adjustbox}
\graphicspath{ {./images/} }
\usepackage[utf8]{inputenc}
\usepackage[english]{babel}
\usepackage{float}
\usepackage{ctex}
\usepackage{lipsum}
\usepackage{graphicx}
\usepackage{float}
\usepackage[margin=0.7in]{geometry}
\usepackage{amsmath}
\usepackage{graphicx}
\usepackage{capt-of}
\usepackage{tcolorbox}
\usepackage{lipsum}
\usepackage{graphicx}
\usepackage{pifont} 
\usepackage{float}
\usepackage{listings}
\usepackage{hyperref} 
\newcommand{\cmark}{\textcolor{green!60!black}{\ding{51}}} % ✓
\newcommand{\xmark}{\textcolor{red}{\ding{55}}}    
\usepackage{xcolor} % For custom colors
\lstset{
	language=Python,                % Choose the language (e.g., Python, C, R)
	basicstyle=\ttfamily\small, % Font size and type
	keywordstyle=\color{blue},  % Keywords color
	commentstyle=\color{gray},  % Comments color
	stringstyle=\color{red},    % String color
	numbers=left,               % Line numbers
	numberstyle=\tiny\color{gray}, % Line number style
	stepnumber=1,               % Numbering step
	breaklines=true,            % Auto line break
	backgroundcolor=\color{black!5}, % Light gray background
	frame=single,               % Frame around the code
}
\usepackage{float}
\usepackage[]{amsthm} %lets us use \begin{proof}
	\usepackage[]{amssymb} %gives us the character \varnothing
	
	\title{Lecture 3, IEOR 4732\\
		\small{Finite Differences\\有限差分法
	}}
	\author{Zongyi Liu}
	\date{Thu, Mar 27, 2025}
	\begin{document}
		\maketitle
		\tableofcontents

\section{日程 (Agenda)}
\begin{itemize}
  \item 泰勒展開
  \item 有限差分法引入
  \begin{itemize}
  \item 顯性 (explicit), 隱性 (implicit), C-N 法 以及多步格式 (multi-step schemes)
  \item 穩定性分析 (stability analysis)
\end{itemize}
  \item 使用有限差分法對衍生品估計
  \item 應用
    \begin{itemize}
  \item 為各種衍生品, 可轉換債券, 員工股票期權等定價
\item 首次穿越時間 (first passage time)
\item 平滑處理
\end{itemize}
\end{itemize}

\section{泰勒展開 (Taylor Expansion)}

\textbf{Definition}

在 $x=a$ 附近的 $f \in C^{\infty}$  

$$
\begin{aligned}
f(x) & =f(a)+(x-a) f^{(1)}(a)+\frac{(x-a)^{2}}{2!} f^{(2)}(a) +\frac{(x-a)^{3}}{3!} f^{(3)}(a)+\ldots \\
& =\sum_{n=0}^{\infty} \frac{(x-a)^{n}}{n!} f^{(n)}(a)
\end{aligned}
$$

$f \in C^{k+1}$ near $x=a$

\[
f(x) = f(a) + (x - a) f^{(1)}(a) + \frac{(x - a)^2}{2!} f^{(2)}(a) + \cdots + \frac{(x - a)^k}{k!} f^{(k)}(a)
\]

\[
{\quad\quad\quad + \frac{(x - a)^{k+1}}{(k+1)!} f^{(k+1)}(\xi)}
\]

對於一些 $\xi \in(x, a)$\\
最後一個項, 即下面單獨這個, 本稱作 remainder, 且它被寫作 Landau notation:  $O\left((x-a)^{k+1}\right)$.

\subsection{導數估值 (Derivative Approximation)}
\textbf{案例: 一階導的向前和向後的差分近似 (Forward and Backward Difference Approximation of 1st Derivative)}

$f \in C^{2} \quad f^{(1)}=?$

$$
\begin{aligned}
& f(x+h)=f(x)+h f^{(1)}(x)+\frac{h^{2}}{2!} f^{(2)}\left(\xi_{1}\right) \\
& f(x-h)=f(x)-h f^{(1)}(x)+\frac{h^{2}}{2!} f^{(2)}\left(\xi_{2}\right) \\
& f^{(1)}(x)=\frac{f(x+h)-f(x)}{h}-\frac{h}{2} f^{(2)}\left(\xi_{1}\right) \\
& \approx \frac{f(x+h)-f(x)}{h} \text { w/ } O(h) \quad \leftarrow \text { forward } \\
& f^{(1)}(x)=\frac{f(x)-f(x-h)}{h}+\frac{h}{2} f^{(2)}\left(\xi_{2}\right) \\
& \approx \frac{f(x)-f(x-h)}{h} \text { w/ } O(h) \quad \leftarrow \text { backward }
\end{aligned}
$$

\textbf{案例: 一階導的中央差分近似(Central Difference Approximation of 1st Derivative)}

$f \in C^{3} \quad f^{(1)}=?$

$$
\begin{gathered}
f(x+h)=f(x)+h f^{(1)}(x)+\frac{h^{2}}{2!} f^{(2)}(x)+\frac{h^{3}}{3!} f^{(3)}\left(\xi_{3}\right) \\
f(x-h)=f(x)-h f^{(1)}(x)+\frac{h^{2}}{2!} f^{(2)}(x)-\frac{h^{3}}{2!} f^{(3)}\left(\xi_{4}\right) \\
f(x+h)-f(x-h)=2 h f^{(1)}(x)+\frac{h^{3}}{3!} f^{(3)}\left(\xi_{3}\right)+\frac{h^{3}}{3!} f^{(3)}\left(\xi_{4}\right) \\
=2 h f^{(1)}(x)+O\left(h^{3}\right)
\end{gathered}
$$

在這裡 central method 最好, 這個其实我之前學過, 在572那個.

$$
\begin{aligned}
f^{(1)}(x) & =\frac{f(x+h)-f(x-h)}{2 h}+O\left(h^{2}\right) \\
& \approx \frac{f(x+h)-f(x-h)}{2 h} \text { w/ } O\left(h^{2}\right)
\end{aligned}
$$

\textbf{案例: 二階導的中央差分近似(Central Difference Approximation of 2nd Derivative)}

$f \in C^{4} \quad f^{(2)}=?$

$$
\begin{gathered}
f(x+h)=f(x)+h f^{(1)}(x)+\frac{h^{2}}{2!} f^{(2)}(x)+\frac{h^{3}}{3!} f^{(3)}(x)+\frac{h^{4}}{4!} f^{(4)}\left(\xi_{5}\right) \\
f(x-h)=f(x)-h f^{(1)}(x)+\frac{h^{2}}{2!} f^{(2)}(x)-\frac{h^{3}}{3!} f^{(3)}(x)+\frac{h^{4}}{4!} f^{(4)}\left(\xi_{6}\right) \\
f(x+h)+f(x-h)=2 f(x)+2 \frac{h^{2}}{2!} f^{(2)}(x)+\frac{h^{4}}{4!} f^{(4)}\left(\xi_{5}\right)+\frac{h^{4}}{4!} f^{(4)}\left(\xi_{6}\right) \\
f^{(2)}(x)=\frac{f(x-h)-2 f(x)+f(x+h)}{h^{2}}+O\left(h^{2}\right)
\end{gathered}
$$

\section{離散化 (Discretization)}
\subsection{熱傳導方程 (Heat Equation)}
start from heat equation (why?)

因為它只涉及一階時間導數和二階空間導數, 易於手工導出顯式與隱式差分格式. 

$$
u_{\tau}-\kappa u_{x x}=0
$$

for $a \leq x \leq b$ and $0 \leq \tau \leq T$

$$
\begin{aligned}
& u(x, 0) =f(x)\\
& u(a, \tau)=g(\tau) \\
& u(b, \tau)=h(\tau)
\end{aligned}
$$

\begin{itemize}
  \item 使一個微分算子 (differential operator) 變為了一個差分方程 (difference equation).
  \item 一個連續的domain $D=\{a \leq x \leq b ; 0 \leq \tau \leq T\}$ 離散化為了 grid.
\end{itemize}

\subsection{網格 (The Grid)}

\includegraphics[max width=0.5\textwidth, center]{grid}
\[
\bar{D} = \left\{
\begin{aligned}
	&x_j = a + (j - 1)\Delta x, &\quad \Delta x = \frac{b - a}{N}, &\quad j = 1, \dots, N+1 \\
	&\tau_k = 0 + (k - 1)\Delta \tau, &\quad \Delta \tau = \frac{T - 0}{M}, &\quad k = 1, \dots, M+1
\end{aligned}
\right\}
\]

然後繼續執行離散化的微分算子

$$
\mathcal{L}(u)=u_{\tau}-\kappa u_{x x}=0
$$

在此之前我們要先使:

\begin{itemize}
  \item $u_{j, k}$ 成為 $u\left(x=x_{j}, \tau=\tau_{k}\right)$ 的實際數值
  \item $U_{j, k}$ 成為其離散近似
\end{itemize}

\section{有限差分法模組 (Finite Differences Schemes)}
\subsection{顯式離散 (Explicit Discretization)}

\includegraphics[max width=0.6\textwidth, center]{2025_03_21_f723913b4d404a19b8e0g-11}

\begin{itemize}
	\item 參考點 (reference point) $\left(x_{j}, \tau_{k}\right)$
	\item 向前 (forward) $\frac{\partial u\left(x_{j}, \tau_{k}\right)}{\partial \tau}$
	\item 中央 (central) $\frac{\partial^{2} u\left(x_{j}, \tau_{k}\right)}{\partial x^{2}}$
\end{itemize}

對於第一個項的向前差分近似: 

\[
\frac{\partial u(x_j, \tau_k)}{\partial \tau}
= \frac{u_{j,k+1} - u_{j,k}}{\Delta \tau} + \mathcal{O}(\Delta \tau)
\]

對於第二個項的中央差分近似: 

\[
\frac{\partial^2 u(x_j, \tau_k)}{\partial x^2}
= \frac{u_{j-1,k} - 2u_{j,k} + u_{j+1,k}}{\Delta x^2} + \mathcal{O}(\Delta x^2)
\]

\[
u_\tau(x_j, \tau_k) - \kappa u_{xx}(x_j, \tau_k) = 0
\]

\[
\frac{u_{j,k+1} - u_{j,k}}{\Delta \tau}
- \kappa \frac{u_{j-1,k} - 2u_{j,k} + u_{j+1,k}}{\Delta x^2}
= \mathcal{O}(\Delta x^2) + \mathcal{O}(\Delta \tau)
\]

使用 \( u_{j,k} \) 的近似值可得:

\[
\frac{U_{j,k+1} - U_{j,k}}{\Delta \tau}
- \kappa \frac{U_{j-1,k} - 2U_{j,k} + U_{j+1,k}}{\Delta x^2} = 0
\]

重新調整順序: 

\[
U_{j,k+1} - U_{j,k}
- \boxed{
	\frac{\kappa \Delta \tau}{\Delta x^2}
	\left( U_{j-1,k} - 2U_{j,k} + U_{j+1,k} \right)
} = 0
\]

給定 \( \rho = \dfrac{\kappa \Delta \tau}{\Delta x^2} \), 則有:

\[
U_{j,k+1}
= \rho U_{j-1,k}
+ (1 - 2\rho) U_{j,k}
+ \rho U_{j+1,k}
\qquad \text{for } \quad 2 \leq j \leq N, \quad 1 \leq k \leq M
\]

\vspace{1em}
\[
j = 1 \quad \text{and} \quad j = N + 1 \quad \text{分別對應邊界條件}
\]


$$
\begin{array}{r}
u\left(x_{1}, \tau_{k}\right)=g\left(\tau_{k}\right) \\
u\left(x_{N+1}, \tau_{k}\right)=h\left(\tau_{k}\right) \\
u\left(x_{j}, \tau_{1}\right)=f\left(x_{j}\right)=f_{j}
\end{array}
$$

對於$\tau_{k}=(k-1) \Delta \tau$的近似PDE解為:

$$
\mathbf{U}_{k}=\left(\begin{array}{c}
U_{2, k} \\
U_{3, k} \\
\vdots \\
U_{N, k}^{2}
\end{array}\right)
$$

使用矩陣格式則可以寫作: 

$$
\mathbf{U}_{k+1}=A_{\text {Explicit }} \mathbf{U}_{k}+\rho\left(\begin{array}{c}
U_{1, k} \\
0 \\
\vdots \\
0 \\
U_{N+1, k}
\end{array}\right)
$$

此處有:

\[
A_{\text{Explicit}} =
\begin{pmatrix}
	1 - 2\rho & \rho         &              &              & \\
	\rho      & 1 - 2\rho    & \rho         &              & \\
	& \ddots       & \ddots       & \ddots       & \\
	&              & \rho         & 1 - 2\rho    & \rho \\
	&              &              & \rho         & 1 - 2\rho
\end{pmatrix}
\]

它的初始條件為:

$$
\mathbf{U}_{1}=\left(\begin{array}{c}
U_{2,1} \\
U_{3,1} \\
\vdots \\
U_{N, 1}
\end{array}\right)=\left(\begin{array}{c}
f_{2} \\
f_{3} \\
\vdots \\
f_{N}
\end{array}\right)
$$

\subsection{隱式離散 (Implicit Discretization)}
\begin{center}
\includegraphics[max width=0.6\textwidth]{2025_03_21_f723913b4d404a19b8e0g-17}
\end{center}

\begin{itemize}
	\item 參考點 (reference point) \( (x_j, \tau_{k+1}) \)
	\item 向後 (backward) $\frac{\partial u(x_j, \tau_{k+1})}{\partial \tau}$
	\item 中央 (central) $\frac{\partial^2 u(x_j, \tau_{k+1})}{\partial x^2}$
\end{itemize}


對於第一個項的向後差分近似:

$$
\frac{\partial u\left(x_{j}, \tau_{k+1}\right)}{\partial \tau}=\frac{u_{j, k+1}-u_{j, k}}{\Delta \tau}+O(\Delta \tau)
$$

對於第二個項的中央差分近似:

$$
\frac{\partial^{2} u\left(x_{j}, \tau_{k+1}\right)}{\partial x^{2}}=\frac{u_{j-1, k+1}-2 u_{j, k+1}+u_{j+1, k+1}}{\Delta x^{2}}+O\left(\Delta x^{2}\right)
$$

$$
u_{\tau}\left(x_{j}, \tau_{k+1}\right)-\kappa u_{x x}\left(x_{j}, \tau_{k+1}\right)=0
$$

$$
\frac{u_{j, k+1}-u_{j, k}}{\Delta \tau}-\kappa \frac{u_{j-1, k+1}-2 u_{j, k+1}+u_{j+1, k+1}}{\Delta x^{2}}=O\left(\Delta x^{2}\right)+O(\Delta \tau)
$$

使用 $u_{j, k}$ 的近似值可得:

\[
\frac{U_{j,k+1} - U_{j,k}}{\Delta \tau}
- \kappa \cdot \frac{U_{j-1,k+1} - 2U_{j,k+1} + U_{j+1,k+1}}{\Delta x^2} = 0
\]

重新調整順序:

\[
U_{j,k+1} - U_{j,k}
- \boxed{
	\frac{\kappa \Delta \tau}{\Delta x^2}
	\left( U_{j-1,k+1} - 2U_{j,k+1} + U_{j+1,k+1} \right)
} = 0
\]

{設} \( \rho = \dfrac{\kappa \Delta \tau}{\Delta x^2} \), 則:

\[
- \rho U_{j-1,k+1}
+ (1 + 2\rho) U_{j,k+1}
- \rho U_{j+1,k+1}
= U_{j,k}
\quad \text{for } \quad 2 \leq j \leq N, \quad 1 \leq k \leq M
\]

\vspace{1em}

\[
j = 1 \quad \text{和} \quad j = N+1 \quad \text{對應邊界條件}
\]

\[
u(x_1, \tau_{k+1}) = g(\tau_{k+1})
\]
\[
u(x_{N+1}, \tau_{k+1}) = h(\tau_{k+1})
\]
\[
u(x_j, \tau_1) = u_{j,1} = f(x_j) = f_j
\]

近似的 PDE solution $\tau=k \Delta \tau$ 以矩陣格式寫, 則為:

$$
A_{\text {Implicit }} \mathbf{U}_{k+1}=\mathbf{U}_{k}+\rho\left(\begin{array}{c}
U_{1, k+1} \\
0 \\
\vdots \\
0 \\
U_{N+1, k+1}
\end{array}\right)
$$

此處有:

\[
A_{\text{Implicit}} =
\begin{pmatrix}
	1 + 2\rho & -\rho      &               &               &        \\
	-\rho     & 1 + 2\rho  & -\rho         &               &        \\
	& \ddots     & \ddots        & \ddots        &        \\
	&            & -\rho         & 1 + 2\rho     & -\rho  \\
	&            &               & -\rho         & 1 + 2\rho
\end{pmatrix}
\]

\subsubsection{算法 (Algorithm)}

$$
\mathbf{U}_{1}=\left(\begin{array}{c}
U_{2,1} \\
U_{3,1} \\
\vdots \\
U_{N, 1}
\end{array}\right)=\left(\begin{array}{c}
f_{2} \\
f_{3} \\
\vdots\\
f_{N}
\end{array}\right)
$$

對於 $k=1: M$\\

\[
A_{\text{Implicit}} U_{k+1} = U_k + \rho 
\begin{pmatrix}
	U_{1,k+1} \\
	0 \\
	\vdots \\
	0 \\
	U_{N+1,k+1}
\end{pmatrix}
\]

End

在時間 $T=\tau_{M+1}$ 的最終解為:

$$
U_{M+1} {=}\left(\begin{array}{c}
U_{2, M+1 }\\
U_{3, M+1} \\
\vdots \\
U_{N, M+1}
\end{array}\right)
$$

可見於書的 106頁, 三對角矩陣解法 (tridiagnoal matrix solver)

\subsection{克蘭克-尼柯爾森方法 (Crank-Nicolson)}
\begin{center}
\includegraphics[max width=0.6\textwidth]{2025_03_21_f723913b4d404a19b8e0g-24}
\end{center}

\begin{itemize}
	\item 參考點 (reference point): \( (x_j, \tau_{k + \frac{1}{2}}) \)
	\item 中央 (central): $
	\frac{\partial u(x_j, \tau_{k + \frac{1}{2}})}{\partial \tau}$
	\item 中央 (central):                                                                                                                                                                                                                                                                                                                                                                                                                                                                                                                                                                                                                                                                                                                                                                                                                                                                                                                                                                                                                                                                                                                                                                                                                                                                                                                                                                                                                                                                                                                                                                                                                                                                                                                                                                                                                                                                    $\frac{\partial^2 u(x_j, \tau_{k + \frac{1}{2}})}{\partial x^2}$
\end{itemize}

\[
\left( A_{\text{Implicit}} + I \right) U_{k+1}
= \left( A_{\text{Explicit}} + I \right) U_k
+ \rho
\begin{pmatrix}
	U_{1,k} + U_{1,k+1} \\
	0 \\
	\vdots \\
	0 \\
	U_{N+1,k} + U_{N+1,k+1}
\end{pmatrix}
\]

\begin{itemize}
  \item  $O\left(\Delta x^{2}\right)+O\left(\Delta \tau^{2}\right)$ 的階.
  \item 注意添加隱式和顯式給出的結果是一樣的. 
\end{itemize}

\subsection{多格式組 (Multi-step Scheme)}

\includegraphics[max width=0.6\textwidth, center]{2025_03_21_f723913b4d404a19b8e0g-26}

\begin{itemize}
	\item 參考點 (reference point): $\left(x_{j}, \tau_{k+2}\right)$
	\item 向後的二階 (backward 2nd order): $\frac{\partial u\left(x_{j}, \tau_{k+2}\right)}{\partial \tau}$
	\item 中央 (central): $\frac{\partial^{2} u\left(x_{j}, \tau_{k+2}\right)}{\partial x^{2}}$
\end{itemize}

對於第一個項的後向差分近似:

\[
\frac{\partial u(x_j, \tau_{k+2})}{\partial \tau}
= \frac{u_{j,k} - 4u_{j,k+1} + 3u_{j,k+2}}{2\Delta \tau}
+ \mathcal{O}((\Delta \tau)^2)
\]

對於第二個項的中央差分近似:

\[
\frac{\partial^2 u(x_j, \tau_{k+2})}{\partial x^2}
= \frac{u_{j-1,k+2} - 2u_{j,k+2} + u_{j+1,k+2}}{\Delta x^2}
+ \mathcal{O}((\Delta x)^2)
\]


\[
u_\tau(x_j, \tau_{k+2}) - \kappa u_{xx}(x_j, \tau_{k+2}) = 0
\]

\[
\frac{u_{j,k} - 4u_{j,k+1} + 3u_{j,k+2}}{2\Delta \tau}
- \kappa \frac{u_{j-1,k+2} - 2u_{j,k+2} + u_{j+1,k+2}}{\Delta x^2}
= 0
\]

\[
\frac{U_{j,k} - 4U_{j,k+1} + 3U_{j,k+2}}{2\Delta \tau}
- \kappa \frac{U_{j-1,k+2} - 2U_{j,k+2} + U_{j+1,k+2}}{(\Delta x)^2}
+ \mathcal{O}((\Delta x)^2) + \mathcal{O}((\Delta \tau)^2) = 0
\]

\[
\frac{U_{j,k} - 4U_{j,k+1} + 3U_{j,k+2}}{2\Delta \tau}
- \kappa \frac{U_{j-1,k+2} - 2U_{j,k+2} + U_{j+1,k+2}}{(\Delta x)^2} = 0
\]

\[
\text{order of } \mathcal{O}((\Delta x)^2) + \mathcal{O}((\Delta \tau)^2)
\]

重新排序:

\[
U_{j,k} - 4U_{j,k+1} + 3U_{j,k+2}
- \frac{2\kappa \Delta \tau}{\Delta x^2}
\left( U_{j-1,k+2} - 2U_{j,k+2} + U_{j+1,k+2} \right) = 0
\]

設:
\[
\hat{\rho} = \frac{2\kappa \Delta \tau}{\Delta x^2}
\]

代入得:
\[
- \hat{\rho} U_{j-1,k+2}
+ (3 + 2\hat{\rho}) U_{j,k+2}
- \hat{\rho} U_{j+1,k+2}
= - U_{j,k} + 4U_{j,k+1}
\]

\[
\text{for } 2 \leq j \leq N, \quad 1 \leq k \leq M - 1
\]


\[
A_{\text{MS}} \, U_{k+2} = -U_k + 4U_{k+1}
+ \hat{\rho}
\begin{pmatrix}
	U_{1,k+2} \\
	0 \\
	\vdots \\
	0 \\
	U_{N+1,k+1}
\end{pmatrix}
\]


\text{where}

\[
A_{\text{MS}} =
\begin{pmatrix}
	3 + 2\hat{\rho} & -\hat{\rho} & 0 & \cdots & 0 \\
	-\hat{\rho} & 3 + 2\hat{\rho} & -\hat{\rho} & \ddots & \vdots \\
	0 & -\hat{\rho} & 3 + 2\hat{\rho} & \ddots & 0 \\
	\vdots & \ddots & \ddots & \ddots & -\hat{\rho} \\
	0 & \cdots & 0 & -\hat{\rho} & 3 + 2\hat{\rho}
\end{pmatrix}
\]



\section{穩定性分析 (Stability Analysis)}
\subsection{基礎 (Basics)}
\begin{itemize}
  \item 收斂的 (convergent): 一個差分模組是收斂的, 如果當 $\Delta \tau$ 和 $\Delta x$ 趨向於0, 在定義域 (domain) 的每一個點上, 有限差分的解收斂於實際的數值. 
  \item 連貫的 (consistent): 區域截斷誤差 (local truncation error), 趨向於0, 當 $\Delta \tau$ 和 $\Delta x$ 趨向於0 (由此引發 Taylor's theorem).
  \item 穩定的 (stable): $\left|U_{i j} -u_{i j}\right|<\delta$, 當 $\Delta \tau$ 和 $\Delta x$ 趨向於0時. 
\end{itemize}

\textbf{分析}
\begin{itemize}
  \item 連貫性: 依據泰勒. 
  \item 穩定性: 不容易
  \begin{itemize}
  \item 特徵值分析
  \item 傅立葉分析
  \item 計算受影響域 (computing the domain of dependence)
\end{itemize}
  \item  Lax-Richtmeyer Equivalence Theorem 指出, 一個連續的有限差分模組是收斂的,  同時一個連貫的 (consistent) 有限差分格式是收斂的, 當且僅當它是穩定的 (stable). 如果我們指出它是穩定的, 則我們展示了它在問題的定義域上任何一處都是收斂的.
  \item 一個穩定的格式總是收斂於真正的 PDE 解. 
\end{itemize}

\subsection{案例 (Example)}
$$
\mathcal{L}(u)=u_{\tau}-\kappa u_{x x}=0
$$

無損於一般性 (WLOG):

$$
\begin{aligned}
& u(a, \tau)=0 \\
& u(b, \tau)=0
\end{aligned}
$$

$$
U_{k+1}=B U_{k}
$$

三個格式:

$$
\begin{array}{ll}
B=A_{\text {Explicit }} & \text { 顯性 } \\
B=A_{\text {Implicit }}^{-1} & \text { 隱性 } \\
B=\left(A_{\text {Implicit }}+I\right)^{-1}\left(A_{\text {Explicit }}+I\right) & \text { 克蘭克-尼科爾森}
\end{array}
$$

穩定性條件的本質在於:任何初始誤差被數值解所放大的程度應該有一個上限。

設 $U_{k}$ 為線性系統的真實解,$\hat{U}_{k}$ 為計算出的解。

定義局部誤差如下:


$$
e_{k} \equiv \hat{U}_{k}-U_{k}
$$

$$
\begin{aligned}
B e_{k} & =B\left(\hat{U}_{k}-U_{k}\right) \\
& =\hat{U}_{k+1}-U_{k}+1 \\
& =e_{k+1}
\end{aligned}
$$

通過推斷:


\begin{equation*}
e_{k+1}=B^{k} e_{1} \tag{1}
\end{equation*}


為了使其穩定:


\begin{equation*}
B^{k} e_{1} \rightarrow 0 \quad \text { for } \quad k \rightarrow \infty \tag{2}
\end{equation*}

設 $\lambda_{i}$ 為 $A$ 的特徵值(e-value)。則 $A$ 的譜半徑(spectral radius)為:

$$
\rho(A)=\max _{i}\left|\lambda_{i}\right|
$$

它與矩陣冪次收斂性的關係密切.

\begin{tcolorbox}[width=\linewidth, colframe=OliveGreen, title=引理]
	設 $A \in \mathbb{R}^{n \times n}$,$\rho(A)$ 為其譜半徑。則:
	
	$$
	\lim _{k \rightarrow \infty} A^{k}=0 \text { if and only if } \rho(A)<1
	$$
	
	對於一個 $N \times N$ 的三對角矩陣 $B$:
	
	$$
	B=\left(\begin{array}{ccccc}
		\alpha & \beta & & & \\
		\gamma & \alpha & \beta & & \\
		& \ddots & \ddots & \ddots & \\
		& & \gamma & \alpha & \beta \\
		& & & \gamma & \alpha
	\end{array}\right)
	$$
	
	它的特徵值為:
	
	$$
	\lambda_{B}^{i}=\alpha+2 \beta\left(\frac{\gamma}{\beta}\right)^{1 / 2} \cos \left(\frac{i \pi}{N+1}\right) \quad i=1, \ldots, N
	$$
\end{tcolorbox}



\subsection{顯性方法的穩定性 (Stability of the Explicit Scheme)}
$\alpha = 1 - 2\rho \quad \text{且} \quad \beta = \gamma = \rho,\quad \text{因此其特徵值為}$


\[
\lambda^i_{A_{\text{Explicit}}} = (1 - 2\rho) + 2\rho \cos\left( \frac{i\pi}{N} \right)
\]
\[
= 1 - 2\rho \left( 1 - \cos\left( \frac{i\pi}{N} \right) \right)
\]
\[
= 1 - 4\rho \sin^2\left( \frac{i\pi}{2N} \right)
\]

\[
\left| 1 - 4\rho \sin^2\left( \frac{i\pi}{2N} \right) \right| < 1 
\Rightarrow 
0 < \rho \sin^2\left( \frac{i\pi}{2N} \right) < \frac{1}{2}
\Rightarrow
0 < \rho < \frac{1}{2}
\]

\bigskip

為了使此格式穩定。由於 \( \rho = \dfrac{\kappa \Delta \tau}{(\Delta x)^2} \),這意味著:

\[
0 < \Delta \tau < \frac{(\Delta x)^2}{2\kappa}
\]

\bigskip

柯朗-弗里德里希-萊維條件 (Courant–Friedrichs–Lewy (CFL) condition), 即條件性穩定指出:


$\alpha = 1 + 2\rho$ 且 $\beta = \gamma = -\rho$,因此 $A^{\text{Implicit}}$ 的特徵值為:

$$
\begin{aligned}
& \lambda_{A_{\text {Implicit }}}^{i}=(1+2 \rho)+2(-\rho) \cos \left(\frac{i \pi}{N}\right) \\
& =1+2 \rho\left(1-\cos \left(\frac{i \pi}{N} \right)\right. \\
& =1+4 \rho \operatorname{sin}^{2}\left(\frac{i \pi}{2 N}\right)
\end{aligned}
$$

$$
\begin{aligned}
& \left|\lambda_{B}^{i}\right|=\left|\frac{1}{1+4 \rho \sin ^{2}\left(\frac{i \pi}{2 N}\right)}\right|<1
\end{aligned}
$$

對於所有$i$ (無條件性穩定) 都正確. 

\section{衍生品估值 (Derivative Approximation)}
\subsection{通過有限差分進行衍生品估值: 廣義情況 (Derivative Approximation by Finite Differences: Generic)}
對於

$f \in C^{\infty}$

$$
f(x+i h)=\sum_{n=0}^{\infty} \frac{(i h)^{n}}{n!} f^{(n)}(x)
$$

$f^{(d)}(x)$ 有 $p$如下的等式:

$$
f^{(d)}(x)=\sum_{i=i_{i}}^{i_{u}} \hat{c}_{i}^{d} f(x+i h)+O\left(h^{p}\right)
$$

$\hat{c}_{i}$ 為未知係數,$i_{u}$ 和 $i_{l}$ 分別是我們近似中向前與向後項的數量,乘上 $\dfrac{h^{d}}{d!}$,並定義 $c_{i} = \hat{c}_{i} \dfrac{h^{d}}{d!}$


$$
\frac{h^{d} f^{(d)}(x)}{d!}+O\left(h^{d+p}\right)=\sum_{i=i_{i}}^{i_{u}} c_{i} f(x+i h)
$$

$$
\begin{aligned}
& \frac{h^{d} f^{(d)}(x)}{d!}+O\left(h^{d+p}\right)=\sum_{i=i_{l}}^{i_{u}} c_{i} f(x+i h)
\end{aligned}
$$

$$
\begin{aligned}
& =\sum_{n=0}^{\infty}\left(\sum_{i=i_{l}}^{i_{u}} c_{i} i^{n}\right) \frac{h^{n}}{n!} f^{(n)}(x) \\
& =\sum_{n=0}^{d+p-1}\left(\sum_{i=i_{l}}^{i_{u}} c_{i} i^{n}\right) \frac{h^{n}}{n!} f^{(n)}(x)+O\left(h^{d+p}\right)
\end{aligned}
$$


或者等價地:

$$
f^{(d)}(x)=\frac{d!}{h^{d}} \sum_{n=0}^{d+p-1}\left(\sum_{i=i_{i}}^{i_{u}} c_{i} f^{n}\right)^{n} \frac{h^{n}}{n!} f^{(n)}(x)
$$

 \href{http://papers.ssrn.com/sol3/papers.cfm?abstract_id=3455397}{此文章},提供了高維情況的擴展. 

\subsection{分析 (Findings and Observations)}
\begin{itemize}
	\item 唯一性成立當且僅當我們將約束條件的數量限制為 $d+p$.
	\item 即 $d+p = i_{u} - i_{l} + 1$.
	\item 對於向前差分,有 $i_{l} = 0$ 且 $i_{u} = d+p-1$.
	\item 對於向後差分,有 $i_{l} = -(d+p-1)$ 且 $i_{u} = 0$.
	\item 對於中央差分,有 $i_{l} = -\frac{1}{2}(d+p-1)$ 且 $i_{u} = \frac{1}{2}(d+p-1)$.
\end{itemize}

\subsection{案例 (Example)}
二階三次導數的向前差分. 

\begin{itemize}
  \item $f^{(3)}(x)$ with $O\left(h^{2}\right)$
  \item $d=3$ and $p=2$
  \item $i_{I}=0$ and $i_{u}=4$
\end{itemize}

$$
\sum_{i=0}^{4} c_{i} i^{n}= \begin{cases}1 & n=3 \\ 0 & n \neq 3\end{cases}
$$

$$
\left[\begin{array}{lllll}
0^{0} & 1^{0} & 2^{0} & 3^{0} & 4^{0} \\
0^{1} & 1^{1} & 2^{1} & 3^{1} & 4^{1} \\
0^{2} & 1^{2} & 2^{2} & 3^{2} & 4^{2} \\
0^{3} & 1^{3} & 2^{3} & 3^{3} & 4^{3} \\
0^{4} & 1^{4} & 2^{4} & 3^{4} & 4^{4}
\end{array}\right]\left[\begin{array}{l}
c_{0} \\
c_{1} \\
c_{2} \\
c_{3} \\
c_{4}
\end{array}\right]=\left[\begin{array}{l}
0 \\
0 \\
0 \\
1 \\
0
\end{array}\right] \Rightarrow\left[\begin{array}{l}
c_{0} \\
c_{1} \\
c_{2} \\
c_{3} \\
c_{4}
\end{array}\right]=\left[\begin{array}{l}
-5 / 12 \\
3 / 2 \\
-2 \\
-1 / 4
\end{array}\right]
$$


$$
\begin{aligned}
f^{(3)}(x) & =\frac{3!}{h^{3}} \sum_{i=0}^{4} c_{i} f(x+i h)+O\left(h^{2}\right) \\
& =\frac{-5 f(x)+18 f(x+h)-24 f(x+2 h)+14 f(x+3 h)-3 f(x+4 h)}{2 h^{3}} \\
& +O\left(h^{2}\right)
\end{aligned}
$$


\end{document}