\documentclass[letterpaper]{article} 
\usepackage[utf8]{inputenc}
\usepackage[T1]{fontenc}
\usepackage{amsmath}
\usepackage{ctex} 
\usepackage{amsfonts}
\usepackage{amssymb}
\usepackage{hyperref}
\usepackage[version=4]{mhchem}
\usepackage{stmaryrd}
\usepackage[dvipsnames]{xcolor}
\colorlet{LightRubineRed}{RubineRed!70}
\colorlet{Mycolor1}{green!10!orange}
\definecolor{Mycolor2}{HTML}{00F9DE}
\usepackage{graphicx}
\usepackage{amsmath}
\usepackage{graphicx}
\usepackage{capt-of}
\usepackage{lipsum}
\usepackage{fancyvrb}
\usepackage{tabularx}
\usepackage{listings}
\usepackage[export]{adjustbox}
\graphicspath{ {./images/} }
\usepackage[utf8]{inputenc}
\usepackage[english]{babel}
\usepackage{float}
\usepackage{lipsum}
\usepackage{graphicx}
\usepackage{float}
\usepackage[margin=0.7in]{geometry}
\usepackage{amsmath}
\usepackage{graphicx}
\usepackage{capt-of}
\usepackage{tcolorbox}
\usepackage{lipsum}
\usepackage{graphicx}
\usepackage{float}
\usepackage{listings}
\usepackage{hyperref} 
\usepackage{xcolor} % For custom colors
\lstset{
	language=Python,                % Choose the language (e.g., Python, C, R)
	basicstyle=\ttfamily\small, % Font size and type
	keywordstyle=\color{blue},  % Keywords color
	commentstyle=\color{gray},  % Comments color
	stringstyle=\color{red},    % String color
	numbers=left,               % Line numbers
	numberstyle=\tiny\color{gray}, % Line number style
	stepnumber=1,               % Numbering step
	breaklines=true,            % Auto line break
	backgroundcolor=\color{black!5}, % Light gray background
	frame=single,               % Frame around the code
}
\usepackage{float}
\usepackage[]{amsthm} %lets us use \begin{proof}
	\usepackage[]{amssymb} %gives us the character \varnothing
	
	\title{Lecture 7, IEOR 4732\\
		\small{Monte Carlo Integration and Stochastic Differential Equations\\
			蒙特卡羅積分和隨機微分方程
		}
	}
	\author{Zongyi Liu}
	\date{Thursday, March 6, 2025}

\begin{document}
		\maketitle
\tableofcontents

\section{日程 (Agenda)}
\begin{itemize}
	\item 布朗橋(Brownian Bridge)
	\item 蒙特卡羅積分(Monte Carlo Integration)
	\item 準蒙地卡羅方法(Quasi Monte Carlo)
	\item 隨機微分方程的數值積分(Numerical Integration of SDEs)
	\item 隨機微分方程的模擬(Simulation of SDEs)
	\begin{itemize}
		\item 擴散過程(diffusion processes)
		\item 純粹跳躍過程(pure jump processes)
	\end{itemize}
\end{itemize}

\section{布朗橋 (Brownian Bridge)}
\subsection{引入 (Introduction)}
為了模擬一個在 $t$ 上的維納過程, 有:

$$
W_{t}-W_{0} \sim \sqrt{t} z \quad \text { where } z \sim \mathcal{N}(0,1)
$$

且知道 $W_{0}=0$, 則它變為:

$$
W_{t} \sim \sqrt{t} z^{2}
$$

\begin{itemize}
	\item 假設我們已經模擬出 $W_{t_{1}}$ 和 $W_{t_{2}}$
	\item 現在希望在區間 $\left[t_{1}, t_{2}\right]$ 內填補點,即在已生成的點 $W_{t_{1}}$ 和 $W_{t_{2}}$ 之間進行插值
	\item 為此,我們使用布朗橋(Brownian bridge),要求其通過 $W_{t_{1}}$ 和 $W_{t_{2}}$ 這兩個已知值

\end{itemize}

\textbf{定義}

布朗橋(Brownian bridge)$x$ 是一個在時間 $t_{1}$ 取值為 $a$,在時間 $t_{2}$ 取值為 $b$ 的隨機過程。在 $t_{1}$ 與 $t_{2}$ 之間,$x$ 的行為類似於布朗運動。布朗橋滿足以下條件:

$$
d x_{t}=\frac{b-x_{t}}{t_{2}-t} d t+d B_{t}, \quad x_{t_{1}}=a
$$

此處 $B_{t}$ 是一個標準布朗運動

\subsection{SDE}
這個 SDE 可以被顯性解出, 為:

$$
x_{t}=a \frac{t_{2}-t}{t_{2}-t_{1}}+b \frac{t-t_{1}}{t_{2}-t_{1}}+\left(t_{2}-\Delta t\right)^{t} \int_{t_{1}}^{t} \frac{d B_{u}}{t_{2}-u}
$$

已知 $ f{ }_{x}$ 的條件分佈是正態的. 

$$
\begin{aligned}
& \mathbb{E}_{E} t_{1} \\
&\left.x_{t}\right)=a+(b-a) \frac{t-t_{1}}{t_{2}-t_{1}} \\
& \operatorname{Var}_{t_{1}}\left(x_{t}\right)=\frac{\left(t_{2}-t\right)\left(t-t_{1}\right)}{t_{2}-t_{1}}
\end{aligned}
$$

\subsection{案例 (Example)}
\begin{itemize}
  \item 在區間 $[0, T]$ 上構造一條布朗橋(Brownian bridge),以計算其在各個時刻 $t_j$ 的取值,其中 $j = 1, \ldots, m-1$
\end{itemize}

$$
0<t_{1}<t_{2}<\cdots<t_{m-1}<t_{m}=T
$$

\begin{itemize}
	\item 首先在 $T = t_{\text{trm}}$ 處生成 $W_T$
	\item 然後利用布朗橋來獲得整條路徑在 $\left\{ t_1, t_2, t_3, \ldots, t_{m-1} \right\}$ 上的取值
	\item 利用已知的 $W_T$ 和 $W_{t_0} = W_0 = 0$,生成 $W_{t_1}$
	\item 接著利用 $W_{t_1}$ 和 $W_T$ 生成 $W_{t_2}$,再利用 $W_{t_2}$ 和 $W_T$ 生成 $W_{t_3}$
	\item 如此構造過程持續進行,直到我們到達 $t_{m-1}$
	\item 離散取樣的布朗運動路徑,是透過在時刻 $T, t_1, t_2, t_3, \ldots, t_{m-1}$ 上依下式確定其取值而生成的:
\end{itemize}


$$
\begin{aligned}
& W_{T}=\sqrt{T}_{z_{1}} \\
& W_{t_{1}}=W_{t_{0}}+\left(W_{T}-W_{t_{0}} \frac{t_{1}-t_{0}}{T-t_{0}}+\operatorname{tare}^{0} \sqrt{\frac{\left(T-t_{1}\right)\left(t_{1}-t_{0}\right)}{T-t_{0}}} z_{2}\right. \\
& W_{t_{2}}=W_{t_{1}}+\left(W_{T-{ }_{-}^{2}} W_{t_{1}}\right) \frac{t_{2}-t_{1}}{T-t_{1}}+\sqrt{\frac{\left(T-t_{2}\right)\left(t_{2}-t_{1}\right)}{T-t_{1}}} z_{3} \\
& W_{t_{m j-1}-1}=W_{t_{m-2}}+\left(W_{T}-W_{t_{m-2}}\right) \frac{t_{m-1}-t_{m-2}}{T-t_{m-2}}+\sqrt{\frac{\left(T-t_{m-1}\right)\left(t_{m-1}-t_{m-2}\right)}{T-t_{m-2}}} z_{m} \\
& \text { where } z_{i} \text { for } i=1, \ldots, m \text { are i.i.d. } \mathcal{N}(0,1)
\end{aligned}
$$

\section{蒙特卡羅積分 (Monte Carlo Integration)}
\subsection{定義 (Definition)}
\begin{itemize}
  \item consider
\end{itemize}

$$
\int_{I_{s}} f(x) d x
$$

此處 $I_{s}$ 是 $s$ 維單位立方體,即 $I_{s} = [0,1] \times \cdots \times [0,1]$


\begin{itemize}
	\item 希望對 $s = 20$ 的情況數值計算該積分
	\item 若每個維度僅使用 10 個離散點,則總共會有 $10^{20}$ 個網格點
	\item 由於維度過高導致積分計算困難,這種現象被稱為「維度詛咒」(curse of dimensionality)
	\item 為克服這一障礙,我們採用蒙地卡羅積分(Monte Carlo integration)方法
	\item 相對於嘗試使用某種數值積分公式來計算該積分
	\item 我們從集合中抽樣 $x_{1}, \ldots, x_{N}$,即 $N$ 個均勻分布的向量,並在這些點上對函數 $f$ 進行評估
	\item 然後計算以下總和,作為對積分的近似:
\end{itemize}


$$
\theta_{N}=\frac{1}{N} \sum_{i=1}^{N} f\left(x_{i}\right) \hat{l}_{s}
$$

其中 $\hat{I}_{s}$ 是積分區域 $I_{s}$ 的體積

\begin{itemize}
  \item $I_{s}=[0,1] \times \cdots \times[0,1]$ 所以它有體積 1
\end{itemize}

\begin{itemize}
  \item 這個總和是積分的估值
\end{itemize}

$$
\int_{I_{s}} f(x) d x \approx \theta_{N}
$$

\begin{itemize}
  \item 大數定理:
\end{itemize}


\[
\lim_{N \to \infty} \theta_N = \lim_{N \to \infty} \frac{1}{N} \sum_{i=1}^{N} f(x_i)\, \hat{I}_s
\]
\[
= \hat{I}_s \, \mathbb{E}(f(x))
\]
\[
= \hat{I}_s \int_{I_s} f(x)\, \frac{1}{\hat{I}_s} \, dx
\]
\[
= \int_{I_s} f(x)\, dx
\]

\begin{itemize}
	\item 利用以下事實:$x_i$ 在 $I_s$ 上服從均勻分佈,其概率密度為 $\frac{1}{\hat{I}_s}$
	
	\item 探討當 $N \uparrow \infty$ 時,$\theta_N$ 收斂到積分值的速度有多快
	
	\item 定義:

\end{itemize}


$$
\begin{aligned}
& \delta_{N} \equiv \int_{I_{s}} f(x) d x-\theta_{N} \\
& =\int_{I_{s}} f(x) d x \ln ^{22^{2}} \frac{P^{n^{2}} N}{N} \sum_{i=1} f\left(x_{i}\right) \widehat{I}_{s} \\
& \text { ) } \frac{1}{N} \sum_{i=1}^{N}\left(\int_{l_{s}} f(x) d x-f\left(x_{i}\right) \widehat{I}_{s}\right) \\
& =\frac{1}{N} \sum_{i=1}^{N}\left(\int_{I_{s}} f(x) \frac{1}{\hat{l}_{s}} d x-f\left(x_{i}\right)\right) \hat{I}_{s} \\
& =\frac{\widehat{l}_{s}}{N} \sum_{i=1}^{N}\left(\int_{I_{s}} f(x) \frac{1}{\hat{l}_{s}} d x-f\left(x_{i}\right)\right)
\end{aligned}
$$

\begin{itemize}
  \item 定義:
\end{itemize}

$$
\begin{aligned}
& \Delta f\left(x_{i}\right)=\int_{I_{s}} f(x) \frac{1}{\hat{l}_{s}} d x-f\left(x_{i}\right) \\
& \delta_{N} \equiv=\frac{N_{s}^{2}}{N} \sum_{i=1}^{n} \Delta f\left(x_{i}\right)
\end{aligned}
$$

\begin{itemize}
  \item $\Delta f\left(x_{i}\right)$ 有為零期望:
\end{itemize}

$$
\mathbb{E}\left(\Delta f\left(x_{i}\right)\right)=0
$$

\begin{itemize}
	\item 此外,$\Delta f\left(x_{i}\right)$ 和 $\Delta f\left(x_{j}\right)$ 是不相關的
\end{itemize}

$$
\mathbb{E}\left(\Delta f\left(x_{i}\right) \Delta f\left(x_{j}\right)\right)=\mathbb{E}\left(\Delta f\left(x_{i}\right)\right) \mathbb{E}\left(\Delta f\left(x_{j}\right)\right)=0
$$

\begin{itemize}
  \item  $\delta_{N}$ 的方差
\end{itemize}

$$
\begin{aligned}
\operatorname{Var}\left(\delta_{N}\right) & =\mathbb{E}\left(\delta_{N}^{2}\right)-\left(\mathbb{E}\left(\delta_{N}\right)\right)^{2} \\
& =\frac{\widehat{l}_{s}^{2}}{N^{2}} \sum_{i=1}^{N} \mathbb{E}\left(\Delta f\left(x_{i}\right)^{2}\right)+0 d s \text { in find } \\
& =\frac{\widehat{l}_{s}^{2}}{N}\left\{\int_{\hat{r}_{s}} f^{2}(x) \frac{1}{l_{s}} d x-\left(\int_{I_{s}} f(x) \frac{1}{\widehat{l}_{s}} d x\right)^{2}\right\}
\end{aligned}
$$

\begin{itemize}
  \item 定義:
\end{itemize}

$$
\sigma^{2}(f)=\int_{I_{s}} f^{2}(x) \frac{1}{\hat{I}_{s}} d x-\left(\int_{I_{s}} f(x) \frac{1}{I_{s}} d x\right)^{2}
$$

\begin{itemize}
  \item 我們得到以下的誤差項的方差:
\end{itemize}

$$
\operatorname{Var}\left(\delta_{N}\right)=\frac{\widehat{l}_{s}^{2}}{N} \sigma^{2}(f)
$$

\begin{itemize}
  \item 收斂率為:
\end{itemize}

$$
\sqrt{\operatorname{Var}\left(\delta_{N}\right)} \approx \frac{1^{N}}{\sqrt{N}}
$$

\begin{itemize}
	\item 從方差的角度可以看出,蒙地卡羅積分如何克服維度詛咒(curse of dimensionality)
	
	\item 對於維度為 $s=20$ 的積分,當 $N = 10^{6}$ 時,其誤差約為 $O\left( \frac{10}{\sqrt{N}} \right)^2 \sim \frac{1}{\sqrt{10^{6}}} = 10^{-3}$

\end{itemize}

\subsection{要點 (Important Remarks)}

\begin{itemize}
	\item 誤差項 $\delta_N$ 的大小與維度 $s$ 無關
	\item 誤差的收斂速率為 $O\left(\frac{1}{\sqrt{N}}\right)$,使得收斂速度相對較慢
	\item 因此應該採用某種方差縮減方法(variance reduction method),這部分將在後續討論
	\item 一個誤差上界如下:
\end{itemize}


$$
\int_{I_{s}} f(x) d x \approx \frac{\hat{l}_{s}}{N} \sum_{i=1}^{N} f\left(x_{i}\right) \pm \frac{\hat{l}_{s}^{2}}{N} \sigma^{2}(f)
$$

if $f(x)$ is constant, $f(x) \equiv c$, then

$$
\begin{aligned}
& \sigma^{2}(f)=\int_{l_{s}} f^{2}(x) \frac{1}{\hat{I}_{s}} d x-\left(\int_{l_{s}} f(x) \frac{1}{\hat{I}_{s}} d x\right)^{2}
\end{aligned}
$$

$$
\begin{aligned}
& \stackrel{\text { ass }}{=} c^{2} \int_{I_{s}} \frac{1}{\hat{I}_{s}} d x-c^{2}\left(\int_{I_{s}} \frac{1}{\hat{I}_{s}} d x\right)^{2} \\
& =c^{2}-c^{2} \\
& =0
\end{aligned}
$$

\begin{itemize}
  \item 也就是說,如果被積函數是常數,那麼我們預期的結果就是該常數與積分區域體積的乘積,顯然不需要實際執行積分
\item $\sigma^{2}(f)$ 的解釋是函數 $f(x)$ 偏離一個常數的程度
\item 如果我們能找到一種變換,將座標 $x$ 轉換為新座標 $y$,使得變換後的函數較為平坦,那麼我們就可以降低蒙地卡羅積分的方差
\item 這正是方差縮減(variance reduction)的核心思想
\end{itemize}

\subsection{準蒙特卡羅法 (Quasi-Monte Carlo Methods)}
\begin{itemize}
  \item 準蒙地卡羅方法(Quasi-Monte Carlo methods)可以被視為傳統蒙地卡羅方法的確定性對應方法
\item 用於評估無解析解的多重積分問題
\item 考慮以下情形:
\end{itemize}

$$
\int_{i_{s}} f(x) d x
$$

在 $s$ 維單位立方體上,即 $I_{s} = [0,1] \times \cdots \times [0,1]$

\begin{itemize}
	\item 在傳統的蒙地卡羅積分中,我們選取一組點 $x_{1}, \ldots, x_{N}$,即一組偽隨機數列,並用以下方式來近似該積分:
\end{itemize}

$$
\theta_{N}=\frac{1}{N} \sum_{i=1}^{N} f\left(x_{i}\right) \widehat{I}_{s}
$$

\begin{itemize}
	\item 在準蒙地卡羅方法中,我們以確定性的方式選取點.
	\item 更具體地說,準蒙地卡羅方法產生一組確定性的點列,這些點在 $s$ 維空間中具有最理想的分布.
	\item 這些確定性的點列被稱為低偏差序列(low-discrepancy sequences).
\end{itemize}


\subsection{案例 (Example)}
$$
\int_{0}^{1} \int_{0}^{1} x^{3}\left(1+y^{2}\right) d x d y y^{2}=\frac{1^{0}}{3}
$$

$$
\frac{1}{N} \sum_{j=1}^{N} U_{j}^{3}\left(1+V_{j}^{2}\right)
$$

此處 $U_{j}$ 和 $V_{j}$ 是 i.i.d. $\mathcal{U}(0,1)$ for $j=1, \ldots, N$


\begin{center}
\includegraphics[max width=0.5\textwidth]{2025_03_07_c33339bc5022ea37b83eg-21}
\end{center}
\captionof{figure}{uniform random variables $0.35029$}


\begin{center}
\includegraphics[max width=0.5\textwidth, center]{2025_03_07_c33339bc5022ea37b83eg-22}
\end{center}
\captionof{figure}{low discrepancy sequence (Halton set) ${0 . 3 3 3 3 3}$}

\section{隨機微分方程的數值積分 (Numerical Integration of SDEs)}
\begin{itemize}
  \item 考慮下列一維普遍化 SDE
\end{itemize}

$$
\begin{aligned}
d X_{t} & =\mu\left(X_{t}, t\right) d t+\sigma\left(X_{t}, t\right) d W_{t t^{2}}^{d s} t_{0} \leq t \leq T \\
X\left(t_{0}\right) & =X_{0}
\end{aligned}
$$

\begin{itemize}
  \item 通過伊藤-泰勒擴展:
\end{itemize}

$$
\begin{aligned}
& X_{t}=X_{t 0}+\mu\left(X_{t_{0}}\right) \int_{t_{0}}^{t} d s+\sigma\left(X_{t_{0}}\right) \int_{t_{0}}^{t} d W(s) \\
& e^{d u c a t}+\frac{1}{2} \sigma\left(X_{t_{0}}\right) \sigma^{\prime}\left(X_{t_{0}}\right)\left(\left[W(t)-W\left(t_{0}\right)\right]^{2}-\left(t-t_{0}\right)\right)+R
\end{aligned}
$$

其中 $R$ 是餘項

\begin{itemize}
	\item 當我們模擬一個隨機微分方程時,實際上是在有限個時間點上生成該 SDE 離散版本的樣本.
\end{itemize}


$$
\hat{X}_{\Delta t}, \hat{X}_{2 \Delta t}, \ldots, \hat{X}_{m \Delta t}
$$

\begin{itemize}
	\item $m$ 表示時間步數,$\Delta t$ 表示時間步長, 假設子區間等距,則有 $\Delta t = \frac{T - 0}{m}$
\end{itemize}


$$
\hat{X}_{t_{1}}, \hat{x}_{t_{2}}, \cos \hat{X}_{t_{j}}^{\hat{t}^{2 \tau}}, \ldots, \hat{X}_{t_{m}}
$$

\begin{itemize}
  \item $t_{j}=t_{0}+j \Delta t=j \Delta t$ 對於 $j =1, \ldots, m$
  \item 當 $\Delta t \rightarrow 0$ 時,我們的離散路徑將會收斂到連續路徑
\item 對於區間 $\left[t_{j}, t_{j+1}\right]$,通過選擇
\end{itemize}

$$
\begin{aligned}
t_{0} & =t_{j} \\
t & =t_{j+1} \\
\Delta t & =t_{j+1}-t_{j} \\
\Delta W_{j} & =W\left(t_{j+1}\right)-W\left(t_{j}\right)
\end{aligned}
$$

我們得到以下的表達:

$$
\begin{aligned}
& X_{t_{j+1}}=X_{t_{j}}+\mu\left(X_{t_{j}}\right) \Delta t+\sigma\left(X_{t_{j}}\right) \Delta W_{j}+\frac{1}{2} \sigma\left(X_{t_{j}}\right) \sigma_{j}^{\prime}\left(X_{t_{j}}\right)\left(\left(\Delta W_{j}\right)^{2}-\Delta t\right)+R
\end{aligned}
$$

\section{隨機微分方程模擬 (Simulating SDEs)}
\begin{itemize}
	\item 歐拉格式(Euler Scheme)
	\item 米爾施泰因格式(Milstein Scheme)
	\item 龍格-庫塔格式(Runge–Kutta Scheme)
\end{itemize}

\subsection{歐拉格式 (Euler Scheme)}
\begin{itemize}
  \item 歐拉格式是用於離散化隨機微分方程(SDEs)最簡單的離散化方案
  \item 保持前3個項會給我們顯性歐拉方法如下:
\end{itemize}

\begin{align*}
	\hat{X}_{t_{j+1}} 
	&= \hat{X}_{t_j} + \mu(\hat{X}_{t_j}, t_j) \Delta t + \sigma(\hat{X}_{t_j}, t_j) \Delta W_j \\
	&= \hat{X}_{t_j} + \mu(\hat{X}_{t_j}, t_j) \Delta t + \sigma(\hat{X}_{t_j}, t_j) \sqrt{\Delta t} Z_j
\end{align*}

此處 $Z_{j}$ 是 i.i.d. $\mathcal{N}(0,1)$

\begin{itemize}
  \item 此近似將漂移項展開到 $O(\Delta t)$ 階,但僅將擴散項展開到 $O(\sqrt{\Delta t})$ 階
\end{itemize}

\subsection{米爾施泰因格式 (Milstein Scheme)}
\begin{itemize}
  \item 米爾施泰因格式在歐拉離散化的基礎上加上第二個擴散項,將擴散項的展開提升至 $O(\Delta t)$ 階.
\item 它是通過保留所有 $O(\Delta t)$ 階的項而得出:
\end{itemize}

\begin{align*}
	\hat{X}_{t_{j+1}} 
	&= \hat{X}_{t_j} + \mu(X_{t_j}, t_j) \Delta t 
	+ \sigma(X_{t_j}, t_j) \Delta W_j 
	+ \frac{1}{2} \sigma(X_{t_j}, t_j) \sigma'(X_{t_j}, t_j) 
	\left[ (\Delta W_j)^2 - \Delta t \right] \\
	&= \hat{X}_{t_j} + \mu(\hat{X}_{t_j}, t_j) \Delta t 
	+ \sigma(\hat{X}_{t_j}, t_j) \sqrt{\Delta t} Z_j 
	+ \frac{1}{2} \sigma(X_{t_j}, t_j) \sigma'(X_{t_j}, t_j) 
	\Delta t (Z_j^2 - 1)
\end{align*}

此處 $\sigma^{\prime}(x, t) = \frac{\partial}{\partial x}(\sigma(x, t))$

\begin{itemize}
  \item 雖然米爾施泰因格式具有更高階的離散精度,但它需要已知波動率函數的一階導數.
\end{itemize}

\subsection{龍格-庫塔格式 (Runge-Kutta Scheme)}

\begin{itemize}
	\item 米爾施泰因格式需要知道波動率函數的一階導數,但該導數可能不可得,或計算代價高昂.
	\item 龍格–庫塔格式通過使用龍格–庫塔近似,避免了對波動率函數一階導數的需求,同時仍能保持相同的精度階數.
\end{itemize}

$$
\begin{aligned}
& \widehat{X}_{i} =X_{i}+\mu\left(X_{i}\right) \Delta t+\sigma\left(X_{i}\right) \sqrt{\Delta t} \\
& X_{i +1}^{}=X_{i}+\mu\left(X_{i}\right) \Delta t+\sigma\left(X_{i}\right) \Delta W_{i} \\
& +\frac{1}{2 \sqrt{\Delta t}}\left[\sigma\left(\widehat{X}_{i}\right)-\sigma\left(X_{i}\right)\right]\left(\left(\Delta W_{i}\right)^{2}-\Delta t\right)
\end{aligned}
$$

\section{差分模型下的隨機微分方程模擬 (Simulating SDEs under Different Models)}
\begin{itemize}
	\item 幾何布朗運動(Geometric Brownian Motion, GBM)
	\item 奧爾施泰因–烏倫貝克過程(Ornstein–Uhlenbeck Process)
	\item 考克斯–英格索爾–羅斯模型(Cox–Ingersoll–Ross, CIR)
	\item 帶有隨機波動率的幾何布朗運動(GBMSA,也稱 Heston 模型)
	\item 方差伽馬模型(Variance Gamma, VG model)
	\item 帶有隨機到達的方差伽瑪模型(Variance Gamma with Stochastic Arrival, VGSA)
\end{itemize}


\subsection{幾何布朗運動 (Geometric Brownian Motion)}
\begin{itemize}
  \item GBM 具有如下的 SDE:
\end{itemize}

$$
d X_{t}=\mu X_{t} d t+\sigma X_{t} d W_{t}
$$

\begin{itemize}
  \item 通過伊藤引理的解:
\end{itemize}

$$
x_{T}=X_{0} \exp \left\{\left(\mu^{\prime} \frac{\sigma^{2}}{2}\right) T+\sigma W_{T}\right\}
$$

\begin{itemize}
  \item 已知 $X_T$ 的分佈,因此我們可以直接模擬 $X_T$,無需進行離散化
\end{itemize}

$$
X_{T}=X_{0} \exp \left\{\left(\mu-\frac{\sigma^{2}}{2}\right) T+\sigma \sqrt{T} Z\right\}
$$

where $Z \sim \mathcal{N}(0,1)$

\subsection{奧爾施泰因-烏倫貝克過程 (Ornstein-Uhlenbeck (OU) Process)}
\begin{itemize}
  \item 奧爾施泰因-烏倫貝克過程具有如下 SDE:
\end{itemize}

$$
d X_{t}=\kappa\left(\theta-X_{t}\right) d t+\sigma d W_{t}
$$

\begin{itemize}
  \item 通過伊藤引理的解:
\end{itemize}

$$
X_{T}=e^{-\kappa T} X_{0,}+e^{2} e^{e^{d}}\left(1-e^{-\kappa T}\right)+\sigma e^{\kappa T} \int_{0}^{T} e^{\kappa s} d W_{s}
$$

\begin{itemize}
  \item 此時 $X_T$ 依賴於整條布朗運動路徑.
\item 然而 $X_T$ 的分佈是已知的,因此我們可以直接模擬 $X_T$,無需對 SDE 進行離散化.
\end{itemize}

\subsection{CIR 過程 (CIR process)}
CIR 過程具有如下的 SDE:


\begin{equation*}
d X_{t}=\kappa\left(\theta(t)-X_{t}\right) d t+\sigma \sqrt{X_{t}} d W_{t}^{n} \tag{1}
\end{equation*}


儘管該隨機微分方程(SDE)沒有明確的解析解,但我們並不一定需要解析解才能確定 $X_T$ 的分佈。在 $\theta(t) = \theta$ 為常數的情況下,我們知道 $X_T$ 服從一個非中心卡方分佈(non-central chi-squared distribution),因此可以較為容易地進行模擬。

然而一旦轉為考慮具有時間變化的 $\theta(t)$ 的 CIR 過程,$X_T$ 的分佈就無法明確獲得。此時模擬 $X_T$ 的一種方法是對該 SDE 進行離散化並進行模擬。

這種無法確定 $X_T$ 分佈的情況是非常常見的,因此在實踐上往往需要對隨機微分方程進行模擬.

\subsection{韋斯切克和 CIR (Vesicek vs. CIR)}
\begin{center}
\includegraphics[max width=\textwidth]{CIR}
\end{center}

\subsection{赫斯頓隨機波動模型 (Heston Stochastic Volatility Model (GBMSA))}
\begin{itemize}
  \item 赫斯頓模型具有如下的 SDE:
\end{itemize}

$$
\begin{aligned}
d S_{t} & =r S_{t} d t+\sqrt{v_{t}} S_{t} d W_{S}(t), \\
d v_{t} & =\kappa\left(\theta-v_{t}\right) d t+\sigma \sqrt{v_{t}} d W_{v}(t),
\end{aligned}
$$

\begin{itemize}
	\item 兩個布朗運動分量 $W_{S}(t)$ 與 $W_{v}(t)$ 具有相關係數 $\rho$
	\item 在給定時間 $s_{\text{s}}$ 的條件下,對於 $t > s$,對方差過程進行歐拉離散化
\end{itemize}


$$
v(t) \stackrel{p}{=} v(s)+\kappa(\theta-v(s)) \Delta t+\sigma \sqrt{v(s)} \sqrt{\Delta t} z_{v}
$$

此處 $\Delta t=t-s$ 有 $z_{v} \sim \mathcal{N}(0,1)$

\begin{itemize}
	\item 可以證明上述離散方案在正機率下可能產生負值
	\item 這正是該方案的主要問題
	\item 文獻中針對此問題提出了若干修正方法
	\item Lord 等人將多種歐拉格式統一於以下框架中:
\end{itemize}

$$
v(t)=f_{1}(v(s))+\kappa\left(\theta-f_{2}(v(s))\right) \Delta t+\sigma \sqrt{f_{3}(v(s))} \sqrt{\Delta t} z_{v}
$$

\begin{itemize}
  \item 所有的函數應滿足:對於 $x \geq 0$,有 $f_{i}(x) = x$,並且 $f_{3}(x) \geq 0$ 對所有 $x$ 成立
\item 被認為最有效的方法是完全截斷格式(full truncation scheme)
\item 該方法選擇:$f_{1}(x) = x,\quad f_{2}(x) = f_{3}(x) = x^{+}$,其中 $x^{+} = \max(x, 0)$
\item 由此得到的離散方案為:
\end{itemize}

$$
v(t)=v(s)+\kappa\left(\theta-v(s)^{+}\right) \Delta t+\sigma \sqrt{v(s)^{+}} \sqrt{\Delta t} z_{v}
$$

\begin{itemize}
	\item 可以選擇直接對價格過程應用歐拉離散化方案,或從其精確分佈中進行模擬
	\item 直接離散化得到以下的歐拉方案:
\end{itemize}

$$
S(t)=S(s)+r S(s) \Delta t+S(s) \sqrt{v(t)+} \sqrt{\Delta t} z_{S}
$$

\begin{itemize}
  \item 另一方法來看, 通過使用伊藤引理可以得到確切解:
\end{itemize}

$$
S(t)=S(s) \exp \left[\int_{s}^{1 v}\left(r-\frac{1}{2} v(u)\right) d u+\int_{s}^{t} \sqrt{v(u)} d W_{S}(u)\right]
$$

\begin{itemize}
  \item 通過取對數並應用歐拉離散化,我們得到:
\end{itemize}

$$
\log (S(t))=\log (S(s))+\left[r-\frac{1}{2} v(s)^{+}\right] \Delta t+\sqrt{v(s)^{+}} \sqrt{\Delta t} z_{S}
$$

where $z_{S}=\rho z_{v}+\sqrt{1-\rho^{2}} z$ with $z \sim \mathcal{N}(0,1)$

\subsection{完全截斷算法 (Full Truncation Algorithm)}
\begin{itemize}
	\item 赫斯頓模型的完全截斷格式(full truncation scheme)可總結如下:
	\item 從標準常態分布中生成一個隨機樣本 $z_v$
	\item 在已知 $v(s)$ 的情況下,計算 $v(t)$
	\item 再從標準常態分布中生成一個隨機樣本 $z$,並設 $z_S^0 = \rho z_v + \sqrt{1 - \rho^2} z$
	\item 在已知 $\log S(s)$ 的情況下,計算 $\log S(t)$
	\item 該方法在較粗時間間隔下會產生偏差的估計
	\item 有研究顯示,至少需使用每年 32 個時間步(time steps per year)才能保證偏差足夠小
\end{itemize}


\section{純粹跳躍模擬 (Simulating Pure Jump)}
\subsection{方差伽馬過程 (Variance Gamma Process)}
方差伽馬過程 $X(t ; \sigma, \nu, \theta)$ 是通過將具有漂移 $\theta$ 和波動率 $\sigma$ 的布朗運動,評估於一個由伽瑪過程 $\gamma(t ; 1, \nu)$ 所給定的隨機時間下而獲得的;該伽馬過程具有單位期望速率和方差速率 $\nu$,如下所示:

$$
X(t ; \sigma, \nu, \theta)=\theta \gamma(t ; \mathbb{d} \mathbb{r}, \nu)+\sigma W(\gamma(t ; 1, \nu))
$$

它具有特徵函數:

$$
\phi(\psi)=\mathbb{E}\left(e^{i u x_{t}}\right)=\left(\frac{1}{1-i u \theta \nu+\sigma^{2} u^{2} \nu / 2}\right)^{t / \nu}
$$

假設股票價格過程由具有參數 $\sigma, \nu, \theta$ 的幾何方差伽瑪(geometric VG)分佈給出,則時間 $t$ 時的對數價格為:

$$
\ln S_{t}=\ln S_{0}+(r-q+\omega) t^{2}+X(t ; \sigma, \nu, \theta)
$$

where

$$
\begin{aligned}
& \\
& =\frac{1}{t} \ln \left(1-\theta \nu-\sigma^{2} \nu / 2\right)
\end{aligned}
$$

除了正態分布的波動率參數 $\sigma$ 外,該模型還包含下列控制參數:\\
(i) 峰度參數 (kurtosis) $\nu$,用於控制「厚尾性」(kurtosis),即報酬分布相較於常態分布在左右兩側尾部機率的對稱性增強;\\
(ii) 偏度參數 (skewness) $\theta$,用於允許報酬密度的左右尾部出現不對稱(skewness)。

該模型另一個吸引人的特點是,它將對數常態分布與 B-S-M 公式作為參數化的特殊情形(當 $\nu=0$ 且 $\theta=0$)所包含其中。

\subsection{方差伽馬過程的採樣 (Sampling VG Process)}
\begin{itemize}
	\item 假設將時間區間劃分為 $N$ 個等距子區間,長度為 $h$,其中 $h = \frac{T}{N}$
	\item 從一個具有期望為 $h$, 方差為 $\nu h$ 的伽瑪分布中取樣
	\item 一個伽瑪過程,其形狀參數為 $\alpha$, 尺度參數為 $\beta$:
\end{itemize}

$$
f(x, \alpha, \beta)=\frac{1}{\Gamma(\alpha) \beta^{\alpha}} x^{\alpha-1} e^{\frac{\gamma}{\gamma} e^{\partial(x)}}
$$

\begin{itemize}
  \item 它的期望和方差分別為:
\end{itemize}

$$
\begin{aligned}
\mu^{\mathrm{omp}} & =\alpha \beta \\
\sigma^{2} & =\alpha \beta^{2}
\end{aligned}
$$

\begin{itemize}
  \item 以方差伽馬過程為例:
\end{itemize}

$$
\begin{aligned}
\alpha \beta & =h \\
\alpha \beta^{2} & =\nu h
\end{aligned}
$$

\begin{itemize}
  \item 則有:
\end{itemize}

$$
\begin{aligned}
\alpha & =\frac{h}{\nu} \\
\beta & =\nu
\end{aligned}
$$

因此對於方差伽馬過程的一個採樣, $X(h, \sigma, \nu, \theta)$, 則為:

$$
\left.\theta g\left(\frac{h}{\nu}, \nu\right) \not \partial^{g^{e^{2}}}\right) \pi \sqrt{g\left(\frac{h}{\nu}, \nu\right) z}
$$

此處 $z \sim \mathcal{N}(0,1)$ 且 $g\left(\frac{h}{\nu}, \nu\right) \sim \operatorname{gamrand}\left(\frac{h}{\nu}, \nu\right)$

\subsection{算法 (Algorithm)}
\begin{itemize}
  \item 一個模擬方差伽馬過程的算法為:
\end{itemize}

$$
\begin{aligned}
\text { for } i= & 1, \ldots, N \\
z & \sim \mathcal{N}(0,1) \\
g & \sim \operatorname{gamrand}(h / \nu, \nu) \\
X_{i} & =\theta g+\sigma \sqrt{g} z \\
\text { end } &
\end{aligned}
$$

\begin{itemize}
  \item 對於股票價格過程的對數有:
\end{itemize}

$$
\text { for } i=1, \ldots \text { of } N
$$

$$
\log S_{i}^{e s}=\log S_{i-1}+(r-q) h+\omega h+X_{i}
$$

end

\begin{itemize}
  \item 此處 $\omega=\frac{1}{\nu} \log \left(1-\theta \nu-\sigma^{2} \nu / 2\right)$
  \item 注意到 $X(t ; \sigma, \nu=0, \theta=0)=\sigma W_{t} \& \omega=-\frac{1}{2} \sigma^{2}$
\end{itemize}

\subsection{比較 GBM 和 VG (GBM vs VG)}
\includegraphics[max width=\textwidth, center]{GBM}

\subsection{含有隨機到達的方差伽馬模型 (Variance Gamma with Stochastic Arrival (VGSA))}
\begin{itemize}
  \item 定義 CIR 過程 $y(t)$ 為對 SDE 的一個微分方程解. 
\end{itemize}

$$
d y_{t}=\kappa\left(\eta-y_{t}\right) d t+\lambda \sqrt{y_{t}} d W_{t}
$$

\begin{itemize}
	\item $\eta$ 是時間變化的長期平均速率
	\item $\kappa$ 是平均回歸速率
	\item $\lambda$ 是時間變化的波動率
	\item 過程 $y(t)$ 是瞬時時間變化率,因此總的時間變化由 $Y(t)$ 給出,其中
\end{itemize}

$$
Y(t)=\int_{0}^{t} y(u) d u
$$

是對於 $Y(t)$ 的特徵函數, 性質如下:

$$
\begin{aligned}
\mathbb{E}\left(e^{i u Y(t)}\right) & =\phi(u, t, y(0), \kappa, \eta, \lambda) \\
& =A(t, u) e^{B(t, u) y(0)}
\end{aligned}
$$

\begin{itemize}
  \item 此處有:
\end{itemize}

$$
\begin{aligned}
& A(t, u)=\frac{\exp \left(\frac{\kappa^{2} \eta t t}{\| \lambda^{2}}\right)}{\left(\cosh (\gamma t / 2)+\frac{\kappa}{\gamma} \sinh (\gamma t / 2)\right)^{2 \kappa \eta / \lambda^{2}}}
\end{aligned}
$$

\begin{itemize}
  \item 以及:
\end{itemize}

$$
\gamma=\sqrt{\kappa^{2}-2 \lambda^{2} i u}
$$

\begin{itemize}
  \item VGSA 過程被定義為:
\end{itemize}

$$
\begin{aligned}
Z(t) & =X_{V G}(Y(t) ; \sigma, \nu, \theta) \\
& =\theta \gamma(Y(t) ; 1, \nu)+\sigma W(\gamma(Y(t) ; 1, \nu))
\end{aligned}
$$

\begin{itemize}
\item $\sigma, \nu, \theta, \kappa, \eta$ 和 $\lambda$ 是定義改過程的6個參數
\item  特徵函數由以下公式給出:
\end{itemize}

$$
\mathbb{E}\left(e^{i u Z_{V G S A}(t)}\right)^{a^{\text {SS }} e^{S^{e}}}=\phi\left(-i \Psi_{V G}(u), t, \frac{1}{\nu}, \kappa, \eta, \lambda\right)
$$

\begin{itemize}
\item 其中 ${ }_{0}$ 為 $Y(t)$ 的特徵函數,而 $\Psi_{VG}$ 則是方差伽瑪過程在單位時間的對數特徵函數,即:
\end{itemize}

$$
\Psi_{V G}(u)=-\frac{1}{\nu} \log \left(1-i u \theta \nu+\sigma^{2} \nu u^{2} / 2\right)
$$

我們以隨機變量定義時間 $t$ 時的股票價格過程:

$$
S(t)=S(0) \frac{e^{(r-q) t+Z(t)}}{\mathbb{E}\left[e^{Z(t)}\right]}
$$

注意到:
$$
\mathbb{E}\left[e^{Z(t)}\right]=\phi\left(-j \Psi_{V G}(-i), t, \frac{1}{\nu}, \kappa, \eta, \lambda\right)
$$

在方差伽瑪(VG)模型中,這等價於 $e^{-\omega t}$。因此,時間 $t$ 時股票價格對數的特徵函數為:


$$\mathbb{E}\left[e^{i u \log S_{t}}\right]=\exp \left(i u\left(\log S_{0}+(r-q) t\right)\right) \times \frac{\phi\left(-i \Psi_{V G}(u), t, \frac{1}{\nu}, \kappa, \eta, \lambda\right)}{\phi\left(-i \Psi_{V G}(-i), t, \frac{1}{\nu}, \kappa, \eta, \lambda\right)^{i u}}$$

\subsection{VGSA 的 特徵方程 (Characteristic function of VGSA)}
\begin{itemize}
\item 為了模擬 VGSA 過程,如同先前一樣,我們假設將時間區間劃分為 $N$ 個等距子區間,每個區間長度為 $h$,其中 $h = \frac{T_e}{N}$。我們希望在每個時間子區間 $h$ 上模擬 VGSA 過程,在時間點 $t + h$ 進行模擬  
\item 我們可以寫為: (有點問題--

\end{itemize}

$$
\begin{aligned}
\Delta Z_{t} & =Z(t)-Z\left(t_{n} h\right) \\
& =\theta \gamma(Y(t t) ; 1, \nu)+\sigma W(\gamma(Y(t) ; 1, \nu)) \\
& -\quad(\theta \gamma(Y(t-h) ; 1, \nu)+\sigma W(\gamma(Y(t-h) ; 1, \nu))) \\
& +\sigma \sqrt{\gamma(Y(t) ; 1, \nu)-\gamma(Y(t-h) ; 1, \nu)} z
\end{aligned}
$$

其中 $z$ 為服從 $\mathcal{N}(0,1)$ 的標準常態分布隨機變量。

具有期望值 $Y(t)$ 和變異數 $Y(t)\nu$ 的 Gamma 過程 $\gamma(Y(t); 1, \nu)$,其形狀參數(shape)與尺度參數(scale)如下:


$$
\begin{aligned}
\alpha & =\frac{Y(t)}{\nu} \\
\beta & =\nu
\end{aligned}
$$

這暗示了:

$\gamma(Y(t) ; 1, \nu)-\gamma(Y(t-h) ; 1, \nu)) \circ n \stackrel{C o m p}{=} \operatorname{gamma}\left(\frac{Y(t)}{\nu}, \nu\right)-\operatorname{gamma}\left(\frac{Y(t-h)}{\nu}, \nu\right)$

$$
=\operatorname{gamma}\left(\frac{Y(t)-Y(t-h)}{\nu}, \nu\right)
$$

根據伽馬過程的加法性質,因此\\

$\Delta Z_{t}=\theta$ gamma $\left(\frac{Y(t)-Y(t-h)}{\nu}, \nu\right)+\sigma \sqrt{\operatorname{gamma}\left(\frac{Y(t)-Y(t-h)}{\nu}, \nu\right)} z$\\
此處有 $Y(t)-Y(t-h)=\int_{t-h}^{t} y(u) d u$

\begin{itemize}
\item 在模擬中,首先對 CIR 過程進行離散化  
\item 對 CIR 過程使用米爾施泰因離散化方法可得:

\end{itemize}

$$
y_{j}=y_{j-1}+\kappa\left(\eta-y_{j-1}\right) h+\lambda \sqrt{y_{j-1}} \sqrt{h} z^{2}+\frac{\lambda^{2}}{4} h\left(z^{2}-1\right)
$$

\begin{itemize}
\item 其中 $y_{j}$ 是 $y\left(t_{j}\right)$ 的一個近似值,且 $t_{j} = jh$,對所有 $j = 0, \ldots, N$,並且 $z \sim \mathcal{N}(0,1)$  
\item 對於區間 $\left(t_{j-\mathbb{1}}, t_{j}^{c}\right)$,新的時鐘(clock)由以下積分給出:
 
 
 以上可能有問題
 
\end{itemize}

$$
\int_{t_{j-1}}^{t_{j}} y(u) d u
$$

$\sigma^{\circ}$ 通過梯形方法我們可得:

$$
\int_{t_{j-1}}^{t_{j}} y(u) d u=\frac{h}{2}\left(y_{j-1}+y_{j}\right)
$$

\subsection{模擬 VGSA (Simulating VGSA)}
為了模擬 VGSA, 我們可以
$$
\begin{aligned}
& \text { for } j=1, \ldots, N \\
& z \sim \mathcal{N}(0,1) \\
& y_{j}=y_{j-1}+\kappa\left(\eta-y_{j-1}\right) h+\lambda \sqrt{y_{j-1}} \sqrt{h} z+\frac{\lambda^{2}}{4} h\left(z^{2}-1\right) \\
& \hat{t}_{j}=\frac{h}{2}\left(y_{j}+y_{j-1}\right) \\
& g=\operatorname{gamrand}\left(\frac{\hat{t}_{j}}{\nu}, v\right) \\
& z \sim \mathcal{N}(0,1) \\
& X_{j}=\theta g+ \sigma \sqrt{g} z
\end{aligned}
$$

end\\

此處 $X_{j}$ 是對於 $\Delta Z_{t_{j}}$ 的模擬.

\textbf{另一種模擬方法}


另一種模擬 VGSA 的方法是:

\begin{align*}
	\text{for } j = 1, \dots, N \\
	&\quad z \sim \mathcal{N}(0, 1) \\
	&\quad y_j = y_{j-1} + \kappa(\eta - y_{j-1}) h + \lambda \sqrt{y_{j-1}} \sqrt{h} z + \frac{\lambda^2}{4} h (z^2 - 1) \\
	&\quad \hat{t}_j = \frac{h}{2} (y_j + y_{j-1}) \\
	&\quad \hat{\sigma} = \sigma \sqrt{\hat{t}_j} \\
	&\quad \hat{\nu} = \frac{\nu}{\hat{t}_j} \\
	&\quad \hat{\theta} = \theta \hat{t}_j \\
	&\quad g \sim \text{gamrand} \left( \frac{1}{\hat{\nu}}, \hat{\nu} \right) \\
	&\quad \tilde{z} \sim \mathcal{N}(0, 1) \\
	&\quad X_j = \hat{\theta} g + \hat{\sigma} \sqrt{g} \tilde{z} \\
	\text{end}
\end{align*}


對於股票價格的對數, 我們有:


\begin{align*}
	\log S_t &= \log S_0 + (r - q)t + Z(t) - \log\left( \mathbb{E}(e^{Z(t)}) \right) \\
	\log S_{t - h} &= \log S_0 + (r - q)(t - h) + Z(t - h) - \log\left( \mathbb{E}(e^{Z(t - h)}) \right)
\end{align*}

減去:

\begin{align*}
	\log S_t &= \log S_{t-h} + (r - q)h + Z(t) - Z(t - h) + \log\left( \mathbb{E}(e^{Z(t - h)}) \right) - \log\left( \mathbb{E}(e^{Z(t)}) \right) \\
	&= \log S_{t-h} + (r - q)h + \Delta Z_t + \log\left( \mathbb{E}(e^{Z(t - h)}) \right) - \log\left( \mathbb{E}(e^{Z(t)}) \right) \\
	&= \log S_{t-h} + (r - q)h + \Delta Z_t + \Delta \omega_t
\end{align*}

因此:

\begin{align*}
	\text{for } j = 1, \dots, N \\
	\Delta \omega_j &= \log\left( \phi\left(-i\Psi_{VG}(-i), (j-1)h, \frac{1}{\nu}, \kappa, \eta, \lambda \right) \right) \\
	&\quad - \log\left( \phi\left(-i\Psi_{VG}(-i), jh, \frac{1}{\nu}, \kappa, \eta, \lambda \right) \right) \\
	\log S_j &= \log S_{j-1} + (r - q)h + \widehat{\Delta \omega}_j + X_j \\
	\text{end}
\end{align*}



\end{document}