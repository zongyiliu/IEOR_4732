\documentclass[letterpaper]{article} 
\usepackage[utf8]{inputenc}
\usepackage[T1]{fontenc}
\usepackage{amsmath}
\usepackage{amsfonts}
\usepackage{amssymb}
\usepackage{array}
\usepackage{booktabs}
\usepackage{hyperref}
\usepackage[version=4]{mhchem}
\usepackage{stmaryrd}
\usepackage[dvipsnames]{xcolor}
\colorlet{LightRubineRed}{RubineRed!70}
\colorlet{Mycolor1}{green!10!orange}
\definecolor{Mycolor2}{HTML}{00F9DE}
\usepackage{graphicx}
\usepackage{amsmath}
\usepackage{graphicx}
\usepackage{capt-of}
\usepackage{lipsum}
\usepackage{fancyvrb}
\usepackage{tabularx}
\usepackage{listings}
\usepackage[export]{adjustbox}
\graphicspath{ {./images/} }
\usepackage[utf8]{inputenc}
\usepackage[english]{babel}
\usepackage{float}
\usepackage{lipsum}
\usepackage{graphicx}
\usepackage{float}
\usepackage[margin=0.7in]{geometry}
\usepackage{amsmath}
\usepackage{graphicx}
\usepackage{capt-of}
\usepackage{tcolorbox}
\usepackage{lipsum}
\usepackage{graphicx}
\usepackage{float}
\usepackage{listings}
\usepackage{hyperref} 
\usepackage{xcolor} % For custom colors
\lstset{
	language=Python,                % Choose the language (e.g., Python, C, R)
	basicstyle=\ttfamily\small, % Font size and type
	keywordstyle=\color{blue},  % Keywords color
	commentstyle=\color{gray},  % Comments color
	stringstyle=\color{red},    % String color
	numbers=left,               % Line numbers
	numberstyle=\tiny\color{gray}, % Line number style
	stepnumber=1,               % Numbering step
	breaklines=true,            % Auto line break
	backgroundcolor=\color{black!5}, % Light gray background
	frame=single,               % Frame around the code
}
\usepackage{float}
\usepackage[]{amsthm} %lets us use \begin{proof}
	\usepackage[]{amssymb} %gives us the character \varnothing
	
	\title{Homework 2, IEOR 4732}
	\author{Zongyi Liu}
	\date{Sat, Mar 8, 2025}
	\begin{document}
		\maketitle
		
\section{Question}
Pricing an up-and-out call (UOC) - Let $w(x, \tau)$ be the value of a derivative security that satisfies the following PIDE:

$$
\begin{array}{r}
\frac{\partial w}{\partial \tau}(x, \tau)-(r-q) \frac{\partial w}{\partial x}(x, \tau)+r w(x, \tau) \\
-\int_{-\infty}^{\infty}\left[w(x+y, \tau)-w(x, \tau)-\frac{\partial w}{\partial x}(x, \tau)\left(e^{y}-1\right)\right] k(y) d y=0
\end{array}
$$

where:

$$
k(y)=\frac{e^{-\lambda_{p} y}}{\nu y^{1+Y}} \mathbf{1}_{y>0}+\frac{e^{-\lambda_{n}|y|}}{\nu|y|^{1+Y}} \mathbf{1}_{y<0}
$$

with:

$$
\begin{aligned}
& \lambda_{p}=\left(\frac{\theta^{2}}{\sigma^{4}}+\frac{2}{\sigma^{2} \nu}\right)^{\frac{1}{2}}-\frac{\theta}{\sigma^{2}} \\
& \lambda_{n}=\left(\frac{\theta^{2}}{\sigma^{4}}+\frac{2}{\sigma^{2} \nu}\right)^{\frac{1}{2}}+\frac{\theta}{\sigma^{2}}
\end{aligned}
$$

and $x=\ln (S)$ and $\tau=T-t$. For an up-and-out call (UOC) option premium, this PIDE must be solved subject to the initial condition:

$$
w(x, 0)=\left(e^{x}-K\right)^{+}
$$

and boundary conditions:

$$
\begin{aligned}
w\left(x_{0}, \tau\right) & =0 \quad \forall \tau \\
w\left(x_{N}, \tau\right)=w(B, \tau) & =0 \quad \forall \tau
\end{aligned}
$$

Use explicit-implicit finite difference scheme covered during the lecture to solve the PIDE. Calculate UOC option premium in this framework for the following parameters: spot price, $S_{0}=\$ 1900$; strike price $K=2000$; upper barrier $B=2200$, risk-free interest rate, $r=0.25 \%$; dividend rate, $q=1.5 \%$; maturity, $T=0.5$ year; $\sigma=25 \%, \nu=0.31$, $\theta=-0.25$, and $Y=0.4$.

\section{Answer}

To solve the partial integro-differential equation (PIDE) for the up-and-out call (UOC) option using an explicit-implicit finite difference scheme, we need to discretize the equation and implement the numerical method.

As given before, the PIDE for the option price \( w(x, \tau) \) is:

\[
\frac{\partial w}{\partial \tau}(x, \tau)-(r-q) \frac{\partial w}{\partial x}(x, \tau)+r w(x, \tau) 
- \int_{-\infty}^{\infty}\left[w(x+y, \tau)-w(x, \tau)-\frac{\partial w}{\partial x}(x, \tau)\left(e^{y}-1\right)\right] k(y) d y=0
\]

and here \( x = \ln(S) \), \( \tau = T - t \).

\( k(y) \) is the Levy density given by:

\[
k(y)=\frac{e^{-\lambda_{p} y}}{\nu y^{1+Y}} \mathbf{1}_{y>0}+\frac{e^{-\lambda_{n}|y|}}{\nu|y|^{1+Y}} \mathbf{1}_{y<0}
\]

with:
\[
\lambda_{p}=\left(\frac{\theta^{2}}{\sigma^{4}}+\frac{2}{\sigma^{2} \nu}\right)^{\frac{1}{2}}-\frac{\theta}{\sigma^{2}}, \quad \lambda_{n}=\left(\frac{\theta^{2}}{\sigma^{4}}+\frac{2}{\sigma^{2} \nu}\right)^{\frac{1}{2}}+\frac{\theta}{\sigma^{2}}
\]

The initial condition is:
\[
w(x, 0) = \left(e^{x} - K\right)^{+}
\]

The boundary conditions are:
\[
w(x_{0}, \tau) = 0, \quad w(x_{N}, \tau) = w(B, \tau) = 0 \quad \forall \tau
\]


We discretize the domain \( x \in [x_{\text{min}}, x_{\text{max}}] \) and \( \tau \in [0, T] \) into a grid:

\begin{itemize}
	\item \( x_i = x_{\text{min}} + i \Delta x \) for \( i = 0, 1, \dots, N \), where \( \Delta x = \frac{x_{\text{max}} - x_{\text{min}}}{N} \)
	\item \( \tau_j = j \Delta \tau \) for \( j = 0, 1, \dots, M \), where \( \Delta \tau = \frac{T}{M} \)
\end{itemize}

Let \( w_i^j = w(x_i, \tau_j) \). Then we use an explicit-implicit scheme, the explicit part will handle the integral term (jump component), and the implicit part handles the diffusion and drift terms.

The implicit discretization of the diffusion and drift terms is:

\[
\frac{w_i^{j+1} - w_i^j}{\Delta \tau} - (r - q) \frac{w_{i+1}^{j+1} - w_{i-1}^{j+1}}{2 \Delta x} + r w_i^{j+1} = 0
\]

This can be rewritten as:

\[
w_i^{j+1} - \frac{\Delta \tau}{2 \Delta x} (r - q) (w_{i+1}^{j+1} - w_{i-1}^{j+1}) + r \Delta \tau w_i^{j+1} = w_i^j
\]

The integral term is discretized explicitly:

\[
\int_{-\infty}^{\infty} \left[w(x+y, \tau) - w(x, \tau) - \frac{\partial w}{\partial x}(x, \tau) (e^y - 1)\right] k(y) dy
\]

Approximate the integral using a quadrature method (e.g., trapezoidal rule) and evaluate it at time \( \tau_j \):

\[
\sum_{k=-M}^{M} \left[w_{i+k}^j - w_i^j - \frac{w_{i+1}^j - w_{i-1}^j}{2 \Delta x} (e^{y_k} - 1)\right] k(y_k) \Delta y
\]

where \( y_k = k \Delta y \).

The boundary conditions are:

\begin{itemize}
	\item At \( x = x_0 \) and \( x = x_N \), set \( w_0^j = w_N^j = 0 \) for all \( j \)
	\item At \( x = \ln(B) \), set \( w(B, \tau) = 0 \)
\end{itemize}


Then we need to initialize the grid; set \( x_{\text{min}} = \ln(S_{\text{min}}) \), \( x_{\text{max}} = \ln(S_{\text{max}}) \), where \( S_{\text{min}} \) and \( S_{\text{max}} \) are chosen to cover the range of interest. Here $x_{\text{min}}$ and $x_{\text{max}}$ define the spatial domain. The upper limit is the barrier in log-space; the lower limit is set far below the spot to capture deep OTM behavior. And $dx$ and $dt$ are the step sizes for space and time.

Then we get implicit PDE by discretizing the log-transformed PDE:
\begin{itemize}
	\item $\mu$ effective drift in log-space
	\item $\alpha,\beta$: coefficients for the second and first derivatives
	\item $Bl,Bu$: off-diagonal (left and right) terms in the tridiagonal system
	\item $Bd$: diagonal term, which includes discounting and diffusion
\end{itemize}

Then compute the integral term, we can get the lower, main, and upper diagonals of the tridiagonal matrix.

Then get the explicit integral term from the jump process, and apply the barrier conditions, where \texttt{W[j, 0] = 0}: lower price boundary (zero option value) and \texttt{W[j, N\textunderscore x] = 0} is the barrier condition, which will be knocked out if price reaches $S=B$. After solving the system back to $t=0$, the final row holds the option values across all $x$, and we can interpolot between nearby values to get the UOC option premium for \( S_0 = 1900 \).

We are given parameters as \( S_0 = 1900 \), \( K = 2000 \), \( B = 2200 \), \( r = 0.25\% \), \( q = 1.5\% \), \( T = 0.5 \) year, \( \sigma = 25\% \), \( \nu = 0.31 \), \( \theta = -0.25 \), \( Y = 0.4 \). And we can plug them into the code to get the result. 

The python code is given as below:

\begin{lstlisting}
     import numpy as np
     import matplotlib.pyplot as plt
     from scipy.interpolate import interp1d
     from scipy.linalg import solve_banded
     
     # Parameters as given in the question
     S0 = 1900
     K = 2000
     B = 2200
     r = 0.0025
     q = 0.015
     sigma = 0.25
     nu = 0.31
     T = 0.5
     
     # Grid settings
     N_x = 200
     N_t = 500
     x_max = np.log(B)
     x_min = np.log(100)  # low enough to approximate zero
     dx = (x_max - x_min) / N_x
     dt = T / N_t
     
     x = np.linspace(x_min, x_max, N_x + 1)
     S = np.exp(x)
     
     # Initial condition
     w = np.maximum(S - K, 0)
     
     # Coefficients for finite difference
     mu = r - q - 0.5 * sigma**2
     alpha = sigma**2 / (2 * dx**2)
     beta = mu / (2 * dx)
     
     Bl = dt * (alpha - beta)
     Bu = dt * (alpha + beta)
     Bd = 1 + dt * (r + 2 * alpha)
     
     #  Tridiagonal matrix coefficients
     lower = -Bl * np.ones(N_x - 1)
     diag = Bd * np.ones(N_x - 1)
     upper = -Bu * np.ones(N_x - 1)
     
     #  Preallocate the solution matrix 
     W = np.zeros((N_t + 1, N_x + 1))
     W[0, :] = w.copy()
     
     #  Dummy jump term (explicit) 
     R = np.zeros(N_x - 1)
     
     # Time stepping
     for j in range(1, N_t + 1):
     rhs = W[j - 1, 1:N_x] + dt / nu * R
     ab = np.zeros((3, N_x - 1))
     ab[0, 1:] = upper[:-1]
     ab[1, :] = diag
     ab[2, :-1] = lower[1:]
     W[j, 1:N_x] = solve_banded((1, 1), ab, rhs)
     
     # Boundary and barrier condition
     W[j, 0] = 0
     W[j, N_x] = 0  # Barrier at log(B)
     
     # Interpolation at S0
     interp_func = interp1d(x, W[-1, :])
     premium = float(interp_func(np.log(S0)))
     print(f"{premium:.6f}")
     
     # Plot the UOC Option Value and Underlying Price
     plt.figure(figsize=(10, 6))
     plt.plot(S, W[-1, :], label='UOC Option Value at t=0', color='orange')
     plt.axvline(B, color='r', linestyle='--', label='Barrier (Knock-out)')
     plt.axvline(K, color='g', linestyle=':', label='Strike')
     plt.axvline(S0, color='b', linestyle='-.', label='Spot Price')
     plt.xlabel('Underlying Asset Price (S)')
     plt.ylabel('Option Value')
     plt.grid(True)
     plt.legend()
     plt.tight_layout()
     plt.show()
\end{lstlisting}


And the printout is:

\begin{minipage}{\linewidth}
	\begin{Verbatim}
     3.918706
	\end{Verbatim}
\end{minipage}

The three prices with the UOC Option Value can be plotted as below:

\includegraphics[max width=0.8\textwidth, center]{plot}
\captionof{figure}{UOC Option Value and Underlying Price}

\end{document}
